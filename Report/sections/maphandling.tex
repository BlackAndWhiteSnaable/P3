\chapter{Maphandling}\label{ch:map}
Pathfinding is generally the process of finding a path from a given starting point ('A')
to a given destination ('B'),
on a given map.

\section{Section}\label{sec:sec}
Like with many others,
is the first step in Dijkstra's algorithm to reduce the map to the necessary minimum.
After this reduction, the map only consists of \emph{nodes} and \emph{edges}.
An edge connects two nodes together and has one integer \emph{travel cost}.
In this integer is stored how much it costs to traverse along that edge,
measured in the metric that should get optimized (in our case distance).
