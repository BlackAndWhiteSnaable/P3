\chapter{Movement}\label{ch:move}
For our robot to be able to show the results of path finding,
it needed to be able to move.

We decided to move only along a simple 2D grid-like structure,
therefore wheels were the easiest solution.
\todo{we need to make this flow nicer into the next section}

\section{Stepper Motors}\label{sec:motors}
A stepper motor is a motor that moves one step at a time, with its step defined by a step angle.

\begin{figure}[ht]
	\centering
	\includegraphics[width=\textwidth]{figures/move/motor1.png}
	\caption{Step Angle}
	\label{fig:angle} 
\end{figure}

Figure \ref{fig:angle} represents a stepper motor that requires 4 steps to complete a 360 degrees rotation. This determines the step angle to be 90 degrees.
The main components of a stepper motor are represented in Figure \ref{fig:main_components}, and they consist of stators, windings(phases), and rotor.
Attached to the output axle is the rotor, depending on the type of motor it can be magnetized.

\begin{figure}[ht]
	\centering
	\includegraphics[width=0.3\textwidth]{figures/move/motor2.png}
	\caption{Main Components}
	\label{fig:main_components}
\end{figure}

By applying a voltage across one of the windings, current will start flowing through it. By using the right-hand rule, the direction of the magnetic flux can be determined. The flux will want to travel through the path that has the least resistance. This determines the rotor to change its position to minimize resistance. This is shown in Figure \ref{fig:flux}.

\begin{figure}[htp]
	\centering
	\subfloat[High Resistance]{%
		\includegraphics[width=0.4\textwidth]{figures/move/motor3.png}%
		}%
	\hfill
	\subfloat[Low Resistance]{%
		\includegraphics[width=0.4\textwidth]{figures/move/motor4.png}
	}
	\caption{Direction of Magnetic Flux}
	\label{fig:flux}
\end{figure}

\subsection{Types of Stepper Motors}
\subsubsection{Permanent Magnet Motor}
This type of stepper motor has a magnetized rotor. Each winding will be subdivided into two, to better understand how to motor functions. Figure \ref{fig:bas_struct} represents the windings, and how they are distributed inside a stepper motor.

\begin{figure}[htp]
    \centering
    \subfloat[Rotor]{%
        \includegraphics[width=0.4\textwidth]{figures/move/motor5.png}%
        }%
    \hfill%
    \subfloat[Winding]{%
        \includegraphics[width=0.4\textwidth]{figures/move/motor6.png}%
        }%
    \caption{Basic Structure of a Motor}
    \label{fig:bas_struct} 
\end{figure}

The resolution of the motor can be improved in two ways, either by increasing the number of pole pairs in the rotor itself, or by increasing the number of phases as shown in Figure \ref{fig:inc_res}.


\begin{figure}[htp]
    \centering
    \subfloat[Increased Pole Pairs]{%
        \includegraphics[width=0.4\textwidth]{figures/move/motor7.png}%
        }%
    \hfill%
    \subfloat[Increased Number of Winding]{%
        \includegraphics[width=0.4\textwidth]{figures/move/motor8.png}%
        }%
	\hfill%
	\caption{Increased resolution}
	\label{fig:inc_res}
\end{figure}

\newpage
To rotate the motor, simply apply a voltage across the windings in a sequence. A full rotation is shown in Figure \ref{fig:stepping_perm_magn}, with the corresponding phases energized.

\begin{figure}[htp]
    \begin{center}
    \subfloat[$1\textsuperscript{st}$ Step]{
        \includegraphics[width=0.2\textwidth]{figures/move/motor9.png}
        }
    \hfill
    \subfloat[$2\textsuperscript{nd}$ Step]{
        \includegraphics[width=0.2\textwidth]{figures/move/motor10.png}
        }
    \hfill
    \subfloat[$3\textsuperscript{rd}$ Step]{
    	\includegraphics[width=0.2\textwidth]{figures/move/motor11.png}
    	}
  	\hfill
  	\subfloat[$4\textsuperscript{th}$ Step]{
  		\includegraphics[width=0.2\textwidth]{figures/move/motor12.png}
  		}
    \end{center}
    
    \begin{center}
    \subfloat[$1\textsuperscript{st}$ Step Winding]{
        \includegraphics[width=0.2\textwidth]{figures/move/motor13.png}
        }
    \hfill
    \subfloat[$2\textsuperscript{nd}$ Step Winding]{
        \includegraphics[width=0.2\textwidth]{figures/move/motor14.png}
        }
    \hfill
    \subfloat[$3\textsuperscript{rd}$ Step Winding]{
    	\includegraphics[width=0.2\textwidth]{figures/move/motor15.png}
    	}
  	\hfill
  	\subfloat[$4\textsuperscript{th}$ Winding]{
  		\includegraphics[width=0.2\textwidth]{figures/move/motor16.png}
  		}
  	\caption{Stepping a Permanent Magnet Motor}
  	\label{fig:stepping_perm_magn}
    \end{center}
\end{figure}

\subsubsection{Variable Reluctance Motor}
This type of motor, uses a rotor that is not magnetized, and has a number of teeth as seen in Figure \ref{fig:var_rel_components}. The windings are configured differently, as depicted in \ref{fig:var_rel_components}(b), all having a common voltage source but separate ground connections.
They usually have 3 or 5 windings.
Greater precision can be achieved by adding more teeth to the rotor.

\begin{figure}[htp]
    \centering%
    \subfloat[Non Magnetized Rotor]{%
        \includegraphics[width=0.3\textwidth]{figures/move/motor17.png}%
        }%
    \hfill%
    \subfloat[Windings]{%
        \includegraphics[width=0.4\textwidth]{figures/move/motor18.png}%
        }%
    \caption{Variable Reluctance Motor Components}%
    \label{fig:var_rel_components}%
\end{figure}
\newpage

To spin the motor, each winding is energized one at a time, and the rotor rotates to minimize reluctance as explained before.
Some of the differences, between this type of stepper motor and the permanent magnet motor, are that, in order to spin the motor in a direction, the windings have to be energized in a reverse sequence as opposed to the direction of the spin, as depicted in Figure \ref{fig:stepping_var_rel}.

In addition, variable reluctance motors have twice the precision of permanent magnet motors with the same amount of windings.
\begin{figure}[htp]
    \begin{center}
    \subfloat{
        \includegraphics[width=0.2\textwidth]{figures/move/motor19.png}
        }
    \hfill
    \subfloat{
        \includegraphics[width=0.2\textwidth]{figures/move/motor20.png}
        }
    \hfill
    \subfloat{
    	\includegraphics[width=0.2\textwidth]{figures/move/motor21.png}
    	}
  	\hfill
  	\subfloat{
  		\includegraphics[width=0.2\textwidth]{figures/move/motor22.png}
  		}
	\end{center}
	\begin{center}
    \subfloat[$1\textsuperscript{st}$ Step]{
        \includegraphics[width=0.2\textwidth]{figures/move/motor23.png}
        }
    \hfill
    \subfloat[$2\textsuperscript{nd}$ Step]{
        \includegraphics[width=0.2\textwidth]{figures/move/motor24.png}
        }
    \hfill
    \subfloat[$3\textsuperscript{rd}$ Step]{
    	\includegraphics[width=0.2\textwidth]{figures/move/motor25.png}
    	}
  	\hfill
  	\subfloat[$4\textsuperscript{th}$ Step]{
  		\includegraphics[width=0.2\textwidth]{figures/move/motor26.png}
  		}
  	\caption{Stepping Variable Reluctance Motor}
  	\label{fig:stepping_var_rel}
  	\end{center}
\end{figure}
\newpage
\subsubsection{Hybrid Stepper Motor}
Hybrid stepper motors borrow characteristics from both previously mentioned types.

Figure \ref{fig:hybrid_components} shows the two the main components of the hybrid stepper motor. On the left side, the stator can be seen consisting of 8 poles. On the right side the rotor. The rotor consists of two sets of teeth, corresponding to the two poles, north and south.
\begin{figure}[h]
	\centering
	\includegraphics[width=\textwidth]{figures/move/motor27.png}
	\caption{Stator and Rotor}
	\label{fig:hybrid_components}
\end{figure}

It is important to notice two additional things. The first, is that the teeth on the rotor are not aligned but are interleaved. The second, is the placement of the stator teeth in respect to those of the rotor. Both can be observed in Figure \ref{fig:stepper}.

\begin{figure}[htp]
    \centering%
    \subfloat[Interleaved Teeth]{%
        \includegraphics[width=0.4\textwidth]{figures/move/motor28.png}%
        }%
    \hfill%
    \subfloat[Stepper Motor Inside]{%
        \includegraphics[width=0.3\textwidth]{figures/move/motor29.png}%
        }%
    \caption{Hybrid Stepper Motor}
    \label{fig:stepper}
\end{figure}
\newpage
Figure \ref{fig:stepper}, the windings with numbers 1 and 5 are completely aligned with the teeth of the rotor. Windings number 3 and 7 are completely unaligned, while the others are half aligned. This results in higher precision and higher torque offered by the hybrid stepper motor, depending on the stepping method used.

Figures \ref{fig:hybrid_first_step},  \ref{fig:hybrid_second_step}, \ref{fig:hybrid_third_step} represent the way this motor operates.

\begin{center}
	\includegraphics[width=0.8\textwidth]{figures/move/motor32}
	\captionof{figure}{First Step}
	\label{fig:hybrid_first_step}
\end{center}
\newpage
By applying a voltage to both windings, the current flow can be controlled, thereby controlling the polarity of each stator pole, thus controlling the direction of the motor. Notice that, initially, poles A and A’ are completely aligned, and poles B and B’ are half aligned. 

\begin{center}
	\includegraphics[width=0.8\textwidth]{figures/move/motor33}
	\captionof{figure}{Second Step}
	\label{fig:hybrid_second_step}
\end{center}

Next step involves changing the direction of the current in winding A by applying a voltage at the other end of the winding. Even though only the current in winding A has been changed, all stator poles are aligned differently. Poles A and A’ are now half aligned, and poles B and B’ are completely aligned.

\begin{center}
	\includegraphics[width=0.8\textwidth]{figures/move/motor34}
	\captionof{figure}{Third Step}
	\label{fig:hybrid_third_step}
\end{center}
\newpage
Now, changing the direction of the current in winding B, changes the polarity of the stator poles B and B', again, determining a change in the alignment of all stator poles. A and A’ are now completely aligned, and stator poles B and B’ are half aligned. The positions of the stator poles now correspond to those of the first step.

\begin{center}
	\includegraphics[width=0.8\textwidth]{figures/move/motor35}
	\captionof{figure}{Forth Step}
	\label{fig:hybrid_forth_step}
\end{center}


Finally, again changing the direction of the current in winding A, determines the rotor to move another step. Notice the alignment of the stator poles. A and A’ are half aligned, while B and B’ are fully aligned. By changing the direction of the current in winding B, the motor arrives in the initial state, thus repeating the sequence.
\newpage

\subsection{Unipolar And Bipolar Stepper Motors}

Another classification of stepper motors, is depending on the way the windings are configured.
Even though, nowadays, almost every stepper is both.
Meaning that unipolar and bipolar, are rather modes in which the stepper motor can be driven.
Exception being, stepper motors which have only four wires coming out of them, corresponding to bipolar stepper motors.
\todo{rewrite from the perspective of bipolar first, then difference to unipolar. Then pros/cons}

Figure \ref{fig:windings} below represent the configuration of the windings in both unipolar and bipolar stepper motors. 


\begin{figure}[htp] 
    \centering
    \subfloat[Bipolar]{
        \includegraphics[width=0.4\textwidth]{figures/move/motor37.png}
        }
    \hfill
    \subfloat[Unipolar]{%
        \includegraphics[width=0.4\textwidth]{figures/move/motor36.png}
        }
    \caption{Winding Configuration}
    \label{fig:windings}
\end{figure}

Unipolar stepper motors allow current flow in only one direction through the winding, so a center wire has been added, that provides a voltage and determines the stator poles to either be of north or south polarity. Unipolar stepper motors are widely used in applications that require high torque at high speed.

Bipolar stepper motors, allow current to flow in both directions through the windings, so the need of a center wire to provide a voltage disappears. However, the circuit needed to drive a bipolar stepper motor is more complicated. Bipolar stepper motors driving boards exist to make it easier for a person to program the motor. Bipolar stepper motors are used in applications that require high torque at low speeds.
\newpage
Figure \ref{fig:unipolar_stepping} better explains the operating method of unipolar stepper motors. 

\begin{figure}[htp] 
    \centering
    \subfloat[$1\textsuperscript{st}$ Step]{
        \includegraphics[width=0.3\textwidth]{figures/move/motor38.png}
        }
    \hfill
    \subfloat[$2\textsuperscript{nd}$ Step]{
        \includegraphics[width=0.3\textwidth]{figures/move/motor39.png}
        }
    \hfill
    \subfloat[$3\textsuperscript{rd}$ Step]{
    	\includegraphics[width=0.3\textwidth]{figures/move/motor40.png}
    	}
  	\hfill
  	\subfloat[$4\textsuperscript{th}$ Step]{
  		\includegraphics[width=0.3\textwidth]{figures/move/motor41.png}
  		}
  	\hfill
  	\subfloat[$5\textsuperscript{th}$ Step]{
  		\includegraphics[width=0.3\textwidth]{figures/move/motor42.png}
  		}
  	\caption{Unipolar Motor Stepping}
  	\label{fig:unipolar_stepping}
\end{figure}

\todo{only keep bipolar in detail, because that's what we are using}
As seen in Figure \ref{fig:unipolar_stepping}, each step only uses half of the winding, determining the polarity of the stator poles. The voltage is always applied at the same two wires.
\newpage
Note that the bipolar configuration as shown in Figure \ref{fig:bipolar_stepping} allows the current to flow in both directions, but the voltage and ground continuously switch positions. This makes bipolar stepper motors a bit more complicated to drive, but as previously stated, motor driver boards simplify the task.

\begin{figure}[htp] 
    \centering
    \subfloat[$1\textsuperscript{st}$ Step]{
        \includegraphics[width=0.3\textwidth]{figures/move/motor43.png}
        }
    \hfill
    \subfloat[$2\textsuperscript{nd}$ Step]{
        \includegraphics[width=0.3\textwidth]{figures/move/motor44.png}
        }
    \hfill
    \subfloat[$3\textsuperscript{rd}$ Step]{
    	\includegraphics[width=0.3\textwidth]{figures/move/motor45.png}
    	}
   	\hfill
   	\subfloat[$4\textsuperscript{th}$ Step]{
  		\includegraphics[width=0.3\textwidth]{figures/move/motor46.png}
  		}
  	\hfill
  	\subfloat[$5\textsuperscript{th}$ Step]{
  		\includegraphics[width=0.3\textwidth]{figures/move/motor47.png}
  		}
  	\caption{Bipolar Motor Spinning}
  	\label{fig:bipolar_stepping}
\end{figure}
%\section{Tri-State Buffer}\label{sec:buffer}

\section{Wheels}\label{sec:wheels}
The robot should be able to move in eight directions from every position.
By using traditional wheels, the robot would need to be able to steer to the desired direction, thus changing orientation.
This would have been a difficult task raising a number of problems.
Our solution is to use omni-wheels instead.
A standard wheel and an omni-wheel are shown in Figure \ref{fig:wheels}.

The key difference between omni-wheels and traditional wheels is their contact area.
For omni-wheels it consists of smaller wheels that are able to move freely sideways,
thus not generating any friction.

By mounting the wheels in pairs, with the shafts crossing at a 90$^\circ$ angle, we are able to move the robot in any direction without needing to change the orientation of the robot.
This is achieved through rotating the pairs as shown in figure \ref{fig:forces}.

\begin{figure}[htp]
	\centering
	\subfloat[Standard Wheel]{
		\includegraphics[width=0.2\textwidth]{figures/move/regular_wheel}
	}
	\hspace{0.2\textwidth}
	\subfloat[Omni-Wheel]{
		\includegraphics[width=0.2\textwidth]{figures/move/omni_wheel}
	}
	\caption{Wheels}
	\label{fig:wheels}
\end{figure}
\begin{figure}[htp]
	\begin{center}
    \subfloat[North]{
    	\includegraphics[width=0.2\textwidth]{figures/move/vector_addition_North}
    }
    \subfloat[East]{
    	\includegraphics[width=0.2\textwidth]{figures/move/vector_addition_East}
    }
    \subfloat[South]{
   		\includegraphics[width=0.2\textwidth]{figures/move/vector_addition_South}
    }
  	\subfloat[West]{
 		\includegraphics[width=0.2\textwidth]{figures/move/vector_addition_West}
	}
	\end{center}
	\begin{center}
    \subfloat[North-East]{
    	\includegraphics[width=0.2\textwidth]{figures/move/wheel_vectors_NE}
    }
    \subfloat[South-East]{
   		\includegraphics[width=0.2\textwidth]{figures/move/wheel_vectors_SE}
    }
    \subfloat[South-West]{
    	\includegraphics[width=0.2\textwidth]{figures/move/wheel_vectors_SW}
    }
  	\subfloat[North-West]{
  		\includegraphics[width=0.2\textwidth]{figures/move/wheel_vectors_NW}
	}
	\end{center}
  	\caption{Forces from Multiple Wheels Added Together}
  	\label{fig:forces}
\end{figure}


It can also be observed that no two opposite motors spin in different directions,
because this would lead to a rotation, which is undesired for us.
This has also made our task of programming the motors more simple.
%section direction control


\section{Motor Driver Boards}\label{sec:driver_boards}
\todo[inline]{make subsection of motors}
We used motor driver boards in order to drive the motors. They provide a simple interface between the microcontroller and the motors, and make for a better alternative than directly driving the motors from the microcontroller.
\begin{figure}[htp]
	\centering
	\includegraphics[scale=0.5]{figures/move/driver_board}
	\caption{Driver Board}
\end{figure}
\section{Direction Control}\label{sec:direction}
To decide the direction of the robot, we had to control which wheels turn what number of steps. 

One option would have been to control each motor individually.
This required four pins for each motor to step the motors,
and precise timing between the four motors.
Imprecise timing could introduce unintended rotation.

We decided to build our own circuit using tri-state buffers instead.
The circuit will be explained after a short explanation of tri-state buffers.
%This would drastically lower the number of pins needed, and would make for a more precise and advanced way of driving the motors.

\subsection{Tri-State Buffer}
\todo{maybe change where (subsection of direction control)}
To achieve the desired movement using as few pins as possible, we decided to use Tri-State buffers.
Fewer pins make it easier to port this part of the robot to a smaller $\mu$C with fewer pins for a final product.

Tri-State buffers provide the possibility of disconnecting parts of the circuit, when not needed.
This allowed us to manipulate the input to the motors dynamically.

A Tri-State buffer can be thought of as a switch. Figure \ref{fig:tristate} better illustrates that concept.
\begin{figure}[htp]
	\begin{center}
	\subfloat[Disabled]{
		\includegraphics[width=0.3\textwidth]{figures/move/motor48}
		}
	\hspace{2cm}
	\subfloat[Enabled]{
		\includegraphics[width=0.3\textwidth]{figures/move/motor49}
		}
	\caption{Tri-State Buffer Switch Analogy}
	\label{fig:tristate}
	\end{center}
\end{figure}


When the buffer is enabled, its output corresponds to its input, either 0 or 1, "High" or "Low".
However when the buffer is a in its third state, its output is disabled, opening the circuit between the buffer and the next component.
That does not mean its output corresponds to a logic “Low", but instead it is in a state of high impedance in which the output is disconnected from the rest of the circuit.

\subsection{Control Circuit}\label{sub:circuit}
\todo{introduce circuit before}
Table \ref{table:directions} explains which pairs need to be activated, and their rotational direction, in order to achieve the desired movement.

\begin{center}
	\begin{tabular}{|l|l|l|}
		\hline
		Direction & Pair A & Pair B	\\
		\hline
		North & forward & forward \\
		East 	& forward & backward \\
		South & backward & backward \\
		West 	& backward & forward \\
		\hline
		North-East & forward & off \\
		South-East & off & backward \\
		South-West & backward & off\\
		North-West & off & forward \\
		\hline
	\end{tabular}
	\begin{tabular}{|l|c|c|c|c|}
		\hline
		Direction & Aon & Aflip & Bon & Bflip \\
		\hline
		North & 1 & 0 & 1 & 0 \\
		East 	& 1 & 0 & 1 & 1 \\
		South & 1 & 1 & 1 & 1 \\
		West 	& 1 & 1 & 1 & 0 \\
		\hline
		North-East & 1 & 0 & 0 & 1 \\
		South-East & 0 & 1 & 1 & 0 \\
		South-West & 1 & 1 & 0 & 1 \\
		North-West & 0 & 1 & 1 & 1 \\
		\hline
	\end{tabular}
	\captionof{table}{Directions}
	\label{table:directions}
\end{center}



\begin{figure}[htp]
	\centering
	\includegraphics[width=0.9\textwidth]{figures/move/direction_choice}
	\caption{Motor Control Circuit}
	\label{fig:mot_ctrl}
\end{figure}

We decided to use 6 tri-state buffers as shown in figure \ref{fig:mot_ctrl}.
