\chapter{Pathfinding}\label{ch:path}
Pathfinding is generally the process of finding a path from a given starting point ('A')
to a given destination ('B'),
on a given map.

There are different approaches to find the best path,
and different qualifiers for 'best'.

We chose to define 'best' as 'shortest time',
because in a real world application time is the most crucial resource in rescue.
In other applications 'best' could also mean shortest distance, least expensive (toll roads),
most convenient or any number of other qualifiers.

Since our robot has approximately equal movement speed in all used directions,
the shortest distance path will be very close to the shortest time path.
Since our robot only operates on a very limited map,
the shortest distance was also easiest to evaluate.

We chose to start implementing Dijkstra's shortest path algorithm,
since it is fairly simple to understand
and can be used as a baseline for better, more complicated algorithms,
like A*.

\section{Dijkstra}
Like with many others,
is the first step in Dijkstra's algorithm to reduce the map to the necessary minimum.
After this reduction, the map only consists of \emph{nodes} and \emph{edges}.
An edge connects two nodes together and has one integer \emph{travel cost}.
In this integer is stored how much it costs to traverse along that edge,
measured in the metric that should get optimized (in our case distance).

A node has a \emph{name}, an integer \emph{travel cost} and a reference to another node \emph{parent}.
The name is used as an identifier


\cite{Madsen2010}, \cite{Oetiker2010} and \cite{Mittelbach2005}.
\missingfigure{We need a figure right here!}