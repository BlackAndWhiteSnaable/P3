\chapter{Inverter}\label{ch:inverter}
\section{Introduction}
After looking at different schematics for the inverter we decided to have a closer look at the H-bridge model.
We chose this model because it seemed to fit our knowledge level of electrical engineering while still giving us the opportunity to learn on it.

\section{General Function}
For a schematic representation of the H-bridge please see figure ~\ref{fig:h-bridge}.
The general idea is to periodically invert the current flowing through the AC Load.
To achieve this the current has to flow either through \textbf{`A'} and \textbf{`D'} or through \textbf{`B'} and \textbf{`C'}.
\hfill
\missingfigure{Here should be a schematic of an H-bridge}\label{fig:h-bridge}
\newpage 
The switches on either side of the load (\textbf{`A'}\&\textbf{`C'} and \textbf{`B'}\&\textbf{`D'}) are not allowed to be closed simultaneously, because this would result in short-circuiting the DC source.

To ensure that this is no happening we investigated switching \textbf{'AD'} with the rising edge of a clock signal and \textbf{'BC'} with the negative edge of the same clock.

We found out that, due to ?propagation/switching? time, there is a short overlap in timing, when all transistors would be conducting (as can be seen in figure ~\ref{fig:switching_delay}.a)

\missingfigure{switching delay figure, see hand drawing}\label{fig:switching_delay}

Our next idea was to use two different signals for \textbf{'AD'} and \textbf{'BC'}, such that the switching delay is taken into consideration. This can be seen in figure ~\ref{fig:switching_delay}.b.

\section{Formatting}
You have to make two returns to have a new paragraph

This is normal text. \textbf{This is bold text!}
\textit{this is italics.}
\emph{This is emphatic.}

\underline{This is underlined.}

"This is in weird quotation marks."

``This is in proper quotation marks.''

`And this is in single quotes.'
