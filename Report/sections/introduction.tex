\chapter{Introduction}\label{ch:introduction}
In this project we talk about path finding as an implementation in rescue robots.
Looking at the complexity of this task we decided to only stick to very specific subtasks.
The task we decided to look into are:
\begin{itemize}
\item Path finding
\item Obstacle detection
\item Victim identification \todo{do we do this?}
\end{itemize}
We decided to delineate from terrain handling, real size, durability a.o. \todo{maybe add more, just to be sure}
In the following chapters we will talk about real life applications of rescue robots \todo{ref chapter},
differentiation into three main task types \todo{layout known, layout changes, layout unknown} \todo{ref chapter},

\section{Examples}
You can also have examples in your document such as in example~\ref{ex:simple_example}.
\begin{example}{An Example of an Example}
  \label{ex:simple_example}
  Here is an example with some math
  \begin{equation}
    0 = \exp(i\pi)+1\ .
  \end{equation}
  You can adjust the colour and the line width in the {\tt macros.tex} file.
\end{example}

\section{How Does Sections, Subsections, and Subsections Look?}
Well, like this
\subsection{This is a Subsection}
and this
\subsubsection{This is a Subsubsection}
and this.

\paragraph{A Paragraph}
You can also use paragraph titles which look like this.

\subparagraph{A Subparagraph} Moreover, you can also use subparagraph titles which look like this\todo{Is it possible to add a subsubparagraph?}. They have a small indentation as opposed to the paragraph titles.

\todo[inline,color=green]{I think that a summary of this exciting chapter should be added.}
