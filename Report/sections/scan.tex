\chapter{Scan}\label{ch:scan}
In an attempt to take it a step further\note{be more specific please - maybe, To check if the map matches the real physical environment...}, we decided to use IR distance sensors. 
Initially, our robot is supposed to find the 
shortest path from A to B, on a map. 
In rescue situations, often, the map provided does not match 
the actual outline, possibly due to different events that 
altered the outline, such as wall collapses, that might 
obstruct the movement in a certain direction.

By using 8 IR sensors, we can detect changes in all directions 
the robot can move in. If a change is detected, this will be
accounted for in the maphandling code.

\todo{do we mention the range of the sensors and how they work? And the actual distance we have to move in a single step??}

\section{Sensor}\label{sec:sensor}
For detecting any changes in the outline of the map, we have 
used analog IR sensors. Our first idea was to use digital
sensors, but unfortunately we couldn't find any. Using analog sen